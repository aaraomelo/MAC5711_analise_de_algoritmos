\documentclass[a4paper]{article}
\usepackage{amsmath}
\usepackage{amsthm}
\usepackage{amssymb}
\usepackage{tikz}
\usepackage[utf8]{inputenc}

\newcommand{\no}{{ \rm O }}
% \newcommand{\lg}{{ \rm O }}


\newenvironment{myproof}[1][\proofname]{%
  \proof[\ttfamily \large #1]%
}{\endproof}
\newtheorem*{prop}{Proposição}
\newtheorem*{deff}{Definição}

\title{MAC5711 ANÁLISE DE ALGORITMOS: LISTA 1}

\date{}
\begin{document}
\maketitle

\section*{Exercício 1.2.}

\begin{itemize}
    \item Usando a definição de notação $no$, prove que
\end{itemize}

\subsection*{Item a}

\begin{itemize}
    \item $n^7-7n^5+10^{\pi/e}n^2+5000=\no(n^7)$;
\end{itemize}

\begin{myproof}[prova]
    Devemos mostrar que, para $n$ suficientemente grande, existe uma constante $c$ tal que, $n^7-7n^5+10^{\pi/e}n^2+5000 \leq cn^7$. De fato, para $n\geq 0$ tem-se que
    \begin{align*}
        n^7-7n^5+10^{\pi/e}n^2+5000 &\leq n^7+7n^7+10^{2}n^7+5000 n^7\\
        &\leq n^7+7n^7+10^{2}n^7+5000 n^7\\
        n^7-7n^5+10^{\pi/e}n^2+5000&\leq 5108 n^7.
    \end{align*}
\end{myproof}
\subsection*{Item b}

\begin{itemize}
    \item $2\left\lceil\frac{n}{5}\right\rceil=\no(n)$;
\end{itemize}

\begin{myproof}[prova]
    Devemos mostrar que, para $n$ suficientemente grande, existe uma constante $c$ tal que, $2\left\lceil\frac{n}{5}\right\rceil \leq cn$. De fato, para $n\geq 0$ tem-se que
    \begin{align*}
        2\left\lceil\frac{n}{5}\right\rceil &\leq 2\left(\frac{n}{5}+1\right)\\
        &\leq\frac{2}{5}n + 2n\\
        2\left\lceil\frac{n}{5}\right\rceil&\leq \frac{12}{5}n.\\
    \end{align*}
\end{myproof}

\section*{Exercício 1.3.}
\begin{itemize}
    \item  Prove ou dê um contraexemplo para cada uma das afirmações abaixo:
\end{itemize}

\subsection*{Item e}

\begin{itemize}
    \item Se $f(n)=\no(g(n))$ então $2^{f(n)}=\no(2^{g(n)})$
\end{itemize}

\begin{myproof}[prova]
    Dado que, para $n$ suficientemente grande, existe uma constante $c_0$ tal que, $f(n)\leq c_0 g(n)$, devemos mostrar que, para $n$ suficientemente grande, existe uma constante $c_1$ tal que, $2^{f(n)}\leq c_1 2^{g(n)}$. De fato, para $n$ suficientemente grande, tem-se que
    \begin{equation*}
        f(n)\leq c_0 g(n)\Rightarrow 2^{f(n)}\leq 2^{c_0 g(n)} \Rightarrow2^{f(n)}\leq 2^{c_0} 2^{g(n)}.
    \end{equation*}
\end{myproof}


\end{document}
